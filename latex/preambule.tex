% ------------------------------------------------------------------------
% packages
% ------------------------------------------------------------------------
% encodage et écriture en français
\usepackage[utf8]{inputenc}
\usepackage[T1]{fontenc}

% police
\usepackage{lmodern}

% package pour définir des couleurs
\usepackage[svgnames]{xcolor}

% listes à puces
\usepackage{enumitem}

% pour faire des simples quote droite
\usepackage{upquote}

% insertion de code
\usepackage{listings}

% insertion de texte brut
\usepackage{verbatim}
%\usepackage{sverb}

% définition des marges
\usepackage[top=1.5cm, bottom=1.5cm, left=1.5cm, right=1.5cm]{geometry}

% \layout pour voir où on en est avec les marges
\usepackage{layout}

% gérer les interlignes
\usepackage{setspace}

% inclure des images
\usepackage{graphicx}

% packages symboles divers
\usepackage{amsmath}
\usepackage{amsfonts}
\usepackage{amssymb}
\usepackage{mathtools}

% insertion théorèmes, preuves, déf...
\usepackage{amsthm}

% tableaux
\usepackage{array}

% gestion espaces en fin de commandes sans argument
\usepackage{xspace}

% pour augmenter la taille des équations
\usepackage{relsize}

% faire des copier coller dans le pdf (je crois)
\usepackage{cmap}

% mettre des légendes et labels dans les graphiques
\usepackage{caption}

% lien hypertextes (vers sites internets)
\usepackage{hyperref}

% génération de texte en latin
\usepackage{lipsum}

% écriture en français suite et fin
\usepackage[frenchb]{babel}



% ------------------------------------------------------------------------
% paramètres
% ------------------------------------------------------------------------
% hyperref package settings
\hypersetup{colorlinks=true}

% définition des couleurs
\definecolor{blue}{RGB}{51,131,255}

% définition des marges
% \doublespacing
\onehalfspacing

% options inclusion code
\lstset{%
  language=c++,%
  basicstyle=\scriptsize,%
  showspaces=false,%
  showstringspaces=false,%
  showtabs=false,%
  numbers=left,%
  numberstyle=\color{gray},%
  rulecolor=\color{black},%
  tabsize=2,%
%title=\lstname,%
  commentstyle=\color{gray},%
  inputencoding=utf8,%
  literate=%
  {á}{{\'a}}1 {é}{{\'e}}1 {í}{{\'i}}1 {ó}{{\'o}}1 {ú}{{\'u}}1%
  {Á}{{\'A}}1 {É}{{\'E}}1 {Í}{{\'I}}1 {Ó}{{\'O}}1 {Ú}{{\'U}}1%
  {à}{{\`a}}1 {è}{{\`e}}1 {ì}{{\`i}}1 {ò}{{\`o}}1 {ù}{{\`u}}1%
  {À}{{\`A}}1 {È}{{\'E}}1 {Ì}{{\`I}}1 {Ò}{{\`O}}1 {Ù}{{\`U}}1%
  {ä}{{\"a}}1 {ë}{{\"e}}1 {ï}{{\"i}}1 {ö}{{\"o}}1 {ü}{{\"u}}1%
  {Ä}{{\"A}}1 {Ë}{{\"E}}1 {Ï}{{\"I}}1 {Ö}{{\"O}}1 {Ü}{{\"U}}1%
  {â}{{\^a}}1 {ê}{{\^e}}1 {î}{{\^i}}1 {ô}{{\^o}}1 {û}{{\^u}}1%
  {Â}{{\^A}}1 {Ê}{{\^E}}1 {Î}{{\^I}}1 {Ô}{{\^O}}1 {Û}{{\^U}}1%
  {œ}{{\oe}}1 {Œ}{{\OE}}1 {æ}{{\ae}}1 {Æ}{{\AE}}1 {ß}{{\ss}}1%
  {ű}{{\H{u}}}1 {Ű}{{\H{U}}}1 {ő}{{\H{o}}}1 {Ő}{{\H{O}}}1%
  {ç}{{\c c}}1 {Ç}{{\c C}}1 {ø}{{\o}}1 {å}{{\r a}}1 {Å}{{\r A}}1%
  {€}{{\EUR}}1 {£}{{\pounds}}1%
}

% espaces entre paragraphes
\setlength{\parskip}{1em}



% ------------------------------------------------------------------------
% macros
% ------------------------------------------------------------------------
% raccourcis mathématiques
\newcommand{\rr}{\ensuremath{\mathbb{R}}\xspace}
\newcommand{\nn}{\ensuremath{\mathbb{N}}\xspace}
\newcommand{\cc}{\ensuremath{\mathbb{C}}\xspace}
\newcommand{\eps}{\ensuremath{\varepsilon}\xspace}
\newcommand{\xin}{\ensuremath{(X_{i})_{i = 1}^{n}}\xspace}
\newcommand{\normale}{\ensuremath{\mathcal{N}(0, 1)}\xspace}

% raccourcis de français
\newcommand{\iid}{indépendantes et identiquement distribuées\xspace}
\newcommand{\va}{variable aléatoire\xspace}
\newcommand{\vas}{variables aléatoires\xspace}
\newcommand{\bs}{bootstrap\xspace}
\newcommand{\ic}{intervalle de confiance\xspace}
\newcommand{\esb}{estimateur sans biais\xspace}
\newcommand{\fdr}{fonction de répartition\xspace}
\newcommand{\lm}{Lax-Milgram}


% macros stat param
\newcommand{\ee}[1]{\mathbb{E}\left( #1\right)}
\newcommand{\pp}[1]{\mathbb{P}\left( #1\right)}
\newcommand{\var}[1]{\mathrm{Var}\left( #1\right)}
\newcommand{\biais}[1]{\mathrm{biais}\left( #1\right)}
\newcommand{\ech}[1]{\ensuremath{(#1_{i})_{i = 1}^{n}}\xspace}
\renewcommand{\hat}{\widehat}
\newcommand*\diff{\mathop{}\!\mathrm{d}}
\renewcommand{\vec}[1]{\ensuremath{\overrightarrow{#1}}}
\renewcommand{\det}[1]{\mathrm{det}\left(#1\right)}

% environnements
\newenvironment{v_proof}%
{\vspace{1em} \itshape \textbf{Démonstration :\\}}%
{$\square$ \upshape \vspace{2em}}

\DeclarePairedDelimiter\abs{\lvert}{\rvert}%
\DeclarePairedDelimiter\norm{\lVert}{\rVert}%

% sectionnement
\makeatletter
\renewcommand{\section}{\@startsection {section}{1}{\z@}%
  {-3.5ex \@plus -1ex \@minus -.2ex}%
  {2.3ex \@plus .2ex}%
{\reset@font\Large\bfseries\color{OrangeRed}\sffamily}}

\renewcommand{\subsection}{\@startsection {subsection}{2}{\z@}%
  {-3.25ex \@plus -1ex \@minus -.2ex}%
  {1.5ex \@plus .2ex}%
{\normalfont\large\bfseries\color{OrangeRed}\sffamily}}

\renewcommand{\subsubsection}{\@startsection {subsubsection}{3}{\z@}%
  {-3.25ex\@plus -1ex \@minus -.2ex}%
  {1.5ex \@plus .2ex}%
{\normalfont\normalsize\bfseries\color{OrangeRed}\sffamily}}

\renewcommand{\paragraph}{\@startsection {paragraph}{4}{\z@}%
  {3.25ex \@plus 1ex \@minus .2ex}%
  {-1em}%
{\normalfont\normalsize\bfseries\sffamily}}

\renewcommand{\subparagraph}{\@startsection {subparagraph}{5}{\parindent}%
  {3.25ex \@plus 1ex \@minus .2ex}%
  {-1em}%
{\normalfont\normalsize \bfseries\sffamily}}
\makeatother



\theoremstyle{remark}
\newtheorem{remark}{Remarque}
\newtheorem{example}{Exemple}
\newtheorem{perspectives}{Perspectives}
\newtheorem{exercise}{Exercice}

\theoremstyle{definition}
\newtheorem{definition}{Définition}[section]

\theoremstyle{plain}
\newtheorem{proposition}[definition]{Proposition}
\newtheorem{property}[definition]{Propriété}
\newtheorem{theorem}[definition]{Théorème}
\newtheorem{corollary}[definition]{Corollaire}
\newtheorem{lemma}[definition]{Lemme}

\renewcommand{\qedsymbol}{$\square$}
